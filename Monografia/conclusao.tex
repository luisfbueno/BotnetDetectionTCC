\chapter{Conclusão}
\label{c.conclusao}

Este estudo fez uma análise sobre a utilização de técnicas da área de \textit{Machine Learning} para a detecção de \textit{Botnets} baseadas no protocolo IRC em uma rede, através da análise fluxos bidirecionais. Estudos deste tipo se fazem cada vez mais relevante tendo em vista o crescimento do uso de aparelhos conectados em rede que podem ser integrados às \textit{botnets}, aumentando ainda mais o poder de seus ataques.

Sendo assim, é possível afirmar que o presente estudo cumpriu seus objetivos ao estudar a aplicação e eficiência de vários algoritmos da área, além de explorar a utilização de características utilizadas em estudos passados e ainda explorar novas possibilidades de uso. 

Com base nos resultados, é possível afirmar que os classificadores testados foram bem sucedidos em suas tarefas, obtendo altos níveis de acurácia. Dentre eles destacam-se os métodos baseados em Árvores, que obtiveram os melhores resultados, seguidos pelos métodos de \textit{Support Vector Machines}, e por último pelos métodos de \textit{Naive Bayes}.

Com relação às duas abordagens de utilização de características, é possível observar que a utilização do RFE para seleção de características impactou diretamente na melhoria da acurácia dos modelos probabilísticos de Naive Bayes, no entanto para os outros métodos os resultados foram parecidos. Porém, a utilização do RFE ainda apresenta menor custo computacional, o que a torna mais efetiva.

Já para o número de características, percebe-se que nenhum deles apresentou uma diferença muito grande com exceção do método de Florestas Aleatórias que acabou utilizando todas as características presentes no conjunto, e assim utilizando 34 características a mais que na abordagem de Força Bruta. Para a primeira abordagem, utilizando 11 características foi obtida uma média de 6,5 características utilizadas por classificador, enquanto para a segunda utilizando 39 características observou-se uma média de 16,8 características.

As características que foram utilizadas nos estudos anteriores se fizeram muito presente nas encontradas pelo RFE, com destaque para algumas que se destacaram em ambas as abordagem: Porcentagem de Pacotes Enviados, IOPR, Média de tamanho do Payload, e o tempo de chegada entre pacotes médio, ou IAT médio. 

Observa-se ainda que no que tange a abordagem do RFE, de maneira geral características ligadas a IAT foram amplamente usadas por quase todos os classificadores, enquanto as características de tamanho de fluxo, como total de bytes, não obtiveram muita representatividade em ambas as abordagens, assim como a característica duração do fluxo.

Por fim, foi possível observar o impacto que a otimização de hiper-parâmetros pode ter no desempenho do SVM, tendo em vista que a troca do valor de seu parâmetro C levou a um considerável aumento em sua acurácia. Além disso, também demonstrou que o Sigmoidal se saiu significativamente pior do que o Linear e o RBF para este tipo de problema.

\section{Trabalhos Futuros}

Como trabalhos futuros baseados neste estudo, têm-se a exploração de outras características possíveis de serem utilizadas na área, avaliar a eficiência dos métodos e características descritos para outros protocolos de \textit{Botnet} centralizadas como o HTTP que vem crescendo em uso recentemente ou descentralizadas que utilizam comunicação P2P, implementação de outras técnicas da área, como Redes Neurais Artificiais ou algoritmos de  ML não-supervisionado para encontrar padrões na rede, trabalhar com outras bases de dados, que apresentem maior volume de pacotes, e implementação de um sistema de monitoramento que analise os fluxos em tempo real.