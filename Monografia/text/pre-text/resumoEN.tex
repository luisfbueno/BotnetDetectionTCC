\begin{resumo}[Abstract]
 \begin{otherlanguage*}{english}
This monograph presents the study of Machine Learning methods applied to the detection of Botnets, compromised computer networks that are controlled by an attacker in order to perform malicious activities such as DDoS attacks, data theft, among others. This work is focused on studying the efficiency of the most used classifiers in previous studies of the aream with the application of Naive Bayes, Support Vector Machines, Decision Trees, Random Forests and AdaBoost models, and apply techniques to select the most relevant network characteristics in the task of selecting botnet traffic in a network environment, through a brute-force approach and using the Recursive Feature Elimination algorithm. It also seeks to study the relevance of optimization techniques on estimators hyper-parameters, in order to increase model accuracy. Finally, conclusions are drawn based on the results obtained in the study. \linebreak
\textbf{Keywords:} Botnet, Machine Learning, Artificial Intelligence, Python.
 \end{otherlanguage*}
\end{resumo}