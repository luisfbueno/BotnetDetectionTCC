\begin{resumo}
Esta monografia apresenta o estudo de métodos da área de \textit{Machine Learning} aplicados para a área de detecção de Botnets, redes de computadores comprometidos que são controlados por um invasor com o fim de executar atividades como ataques \textit{DDoS}, roubos de dados, entre outras ações maliciosas. O presente trabalho se foca em estudar a eficiência dos classificadores mais utilizados em estudos anteriores da área, utilizando as técnicas de \textit{Naive Bayes}, \textit{Support Vector Machines}, Árvores de Decisão, Florestas Aleatórias, \textit{AdaBoost}, e aplicar técnicas para seleção de características de rede mais relevantes na tarefa de seleção dos tráfegos de Botnet em um ambiente de rede, através de uma abordagem de força bruta e uma abordagem utilizando o algoritmo de seleção de características \textit{Recursive Feature Elimination}. Busca também estudar a relevância de técnicas de otimização de hiper-parâmetros dos estimadores, com o objetivo de aumentar a acurácia. Por fim, são apresentadas conclusões com base nos resultados obtidos no estudo. \linebreak
\textbf{Palavras-chave:} Botnet, Machine Learning, Inteligência Artificial, Python.
\end{resumo}